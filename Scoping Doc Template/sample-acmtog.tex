%% This template is based on the sample-acmtog template from overleaf.

%% The first command in your LaTeX source must be the \documentclass command.
\documentclass[acmtog, nonacm]{acmart}

%% \BibTeX command to typeset BibTeX logo in the docs
\AtBeginDocument{%
  \providecommand\BibTeX{{%
    \normalfont B\kern-0.5em{\scshape i\kern-0.25em b}\kern-0.8em\TeX}}}

\setcopyright{none}

%%\acmSubmissionID{123-A56-BU3}

%%
%% For managing citations, it is recommended to use bibliography
%% files in BibTeX format.
%%
%% You can then either use BibTeX with the ACM-Reference-Format style,
%% or BibLaTeX with the acmnumeric or acmauthoryear sytles, that include
%% support for advanced citation of software artefact from the
%% biblatex-software package, also separately available on CTAN.
%%
%% Look at the sample-*-biblatex.tex files for templates showcasing
%% the biblatex styles.
%%

%%
%% The majority of ACM publications use numbered citations and
%% references.  The command \citestyle{authoryear} switches to the
%% "author year" style.
%%
%% If you are preparing content for an event
%% sponsored by ACM SIGGRAPH, you must use the "author year" style of
%% citations and references.
\citestyle{acmauthoryear}

%%
%% end of the preamble, start of the body of the document source.
\begin{document}

%% The "title" command has an optional parameter,
%% allowing the author to define a "short title" to be used in page headers.
\title{Project Title.}
\subtitle{Scoping document for COMP5530}

%% The "author" command and its associated commands are used to define
%% the authors and their affiliations.
%% Of note is the shared affiliation of the first two authors, and the
%% "authornote" and "authornotemark" commands
%% used to denote shared contribution to the research.
\author{Group Member One}
\authornote{Sharing first authorship}
\email{email@leeds.ac.uk}
\author{Group Member Two}
\authornotemark[1]
\email{email@leeds.ac.uk}
\author{Group Member Three}
\authornotemark[1]
\email{email@leeds.ac.uk}
\affiliation{%
  \institution{University of Leeds}
  \streetaddress{Woodhouse}
  \city{Leeds}
  \state{West Yorkshire}
  \country{United Kingdom}
  \postcode{LS2 9BW}
}

\author{Supervisor Name}
\email{shaunpervysier@leeds.ac.uk}
\affiliation{%
  \institution{University of Leeds}
  \country{United Kingdom}}
\additionalaffiliation{the role of Supervisor}

%% By default, the full list of authors will be used in the page
%% headers. Often, this list is too long, and will overlap
%% other information printed in the page headers. This command allows
%% the author to define a more concise list
%% of authors' names for this purpose.
%\renewcommand{\shortauthors}{Dos Anjos, et al.}


%%
%% The code below is generated by the tool at http://dl.acm.org/ccs.cfm.
%% Please copy and paste the code instead of the example below.
%%
\begin{CCSXML}
<ccs2012>
   <concept>
       <concept_id>10003752.10003809.10010052</concept_id>
       <concept_desc>Theory of computation~Parameterized complexity and exact algorithms</concept_desc>
       <concept_significance>100</concept_significance>
       </concept>
   <concept>
       <concept_id>10010583.10010750</concept_id>
       <concept_desc>Hardware~Robustness</concept_desc>
       <concept_significance>300</concept_significance>
       </concept>
   <concept>
       <concept_id>10010147.10010371.10010396.10010401</concept_id>
       <concept_desc>Computing methodologies~Volumetric models</concept_desc>
       <concept_significance>500</concept_significance>
       </concept>
 </ccs2012>
\end{CCSXML}

\ccsdesc[100]{Theory of computation~Parameterized complexity and exact algorithms}
\ccsdesc[300]{Hardware~Robustness}
\ccsdesc[500]{Computing methodologies~Volumetric models}

%%
%% Keywords. The author(s) should pick words that accurately describe
%% the work being presented. Separate the keywords with commas.
\keywords{bread, yeast, baking, food}

\received{20 February 2007}
\received[revised]{12 March 2009}
\received[accepted]{5 June 2009}

%%
%% This command processes the author and affiliation and title
%% information and builds the first part of the formatted document.
\maketitle

%%
\section{Introduction}
\textcolor{blue}{This section provides an overview for the reader to appreciate the level of challenge posed by the project and its relevance to your degree programme. 
This also demonstrates that you have explored the problem space for the project and have acquired an understanding of the initial requirements.
For a strong start, this section subsections should reflect a systematic and scholarly approach to research or literature review you have undertaken to date.
Don't be afraid to cite relevant papers at this point \cite{graphics}.
Some key things that will most likely be here are:
\begin{itemize}
    \item Context: do not drop the problem and goal out of nowhere. \textbf{Why} is this relevant? Motivation, background drivers, social and/or technical context, stakeholders involved.
    \item Problem statement: a very clear problem statement. It can be a hypothesis, a question that needs to be answered, a statement of the current limitations of existing systems, etc. It can take different forms depending on your project.
    \item Possible solution: Initial thoughts on possible solution(s) to be tackled in the project - what may or may not be pursued depending on speed of progress, challenges and how to approach the development of a potential solution, which modules or computing topics will be the building blocks for the solution.
    Should cite background reading or systematic research conducted so far to provide justification of your initial thoughts.
    \item Validation: If it works, what will you be able to do/prove/see? In essence, how to judge the success of your solution in solving the problem. This will help shaping manageable deliverables. 
\end{itemize}
Write it as text so you start getting some practice of how to write your future introduction and abstract.
}

\section{Scope for this project}
\textcolor{blue}{This section specifies what the project will deliver. It should be written in a concise manner, to be used as a basis for assessment. Start with a general overview of what type of project you are working on.}

\subsection{Aim}
\textcolor{blue}{The aim of the project is the overall top-level goal. It might be helpful to consider this in conjunction with the project title.}

\subsection{Objectives}
\textcolor{blue}{List up to five objectives. When you phrase an objective, think about how you can demonstrate its achievement.
To summarise, characteristics of suitable objectives are: 
\begin{itemize}
    \item Deliverable -- you will hand them in! 
    \item Measurable -- examiners are able to judge/quantify if you have done a good job.
    \item Appropriate -- they should solve a sufficiently difficult problem.
    \item Agreed – by your supervisor, assessor, other members of the School (where appropriate).
\end{itemize}
}
\subsection{Deliverables}
\textcolor{blue}{These are items for assessment under ‘delivery’. 
These could be written up as sections in the final project report (e.g. comparison of algorithms, or feasibility assessment, or design documentation, etc.) or be handed in separately (e.g. code, user manual or installation guide etc.). It is important to have the agreement from the supervisor at this stage that these deliverables are suitable as delivery for the type of project. 
Don't need to mention the mandatory deliverables for this module. 
To ensure that these deliverables are within the scope of the project, cross-referencing to the objectives may be a helpful check. 
It may be good then to number your objectives so you can easily relate them here at this section. 
Some examples of deliverables: an analysis of current infrastructure, a comparative study of techniques or tools, a recommendation to the client, requirements specification, design documentation, algorithms, software functionality, a qualitative or quantitative evaluation study, and so on, as appropriate for the type of project.}

\section{Project schedule}
\textcolor{blue}{Talk shortly about how you came up with the plan that you are about to present. 
The schedule for completion of the project should relate the activities (or tasks)  to objectives or deliverables. A few milestones should be identified for self monitoring of progress.}

\subsection{Methodology}
\textcolor{blue}{Outline the underpinning project approach that is appropriate for the chosen type of project. This should help to plan for the order of the activities/tasks}


\subsection{Tasks, milestones and timeline}
\textcolor{blue}{Any appropriate method of presentation is acceptable. A common method is the use of Gantt chart.}

\subsection{Risk assessment}
\textcolor{blue}{If there is any risk identified at this stage (e.g. availability of stakeholders, technical issues or suitable test data etc.), mitigating strategy should be discussed.
You should carefully consider which parts of your project may not be as straightforward, and devise backup plans. }

%%
%% The next two lines define the bibliography style to be used, and
%% the bibliography file.
\bibliographystyle{ACM-Reference-Format}
\bibliography{sample-base}

\appendix

\section{How ethical issues are addressed}
\textcolor{blue}{This is a University requirement. See Resources on 'Ethics relevant to computing projects' for guidance and discuss it with your supervisor. If no ethical issue is involved, a sentence to that effect will suffice.}

\end{document}
\endinput
%%
%% End of file `sample-acmtog.tex'.
